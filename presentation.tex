%% LaTeX Beamer presentation template (requires beamer package)
%% see http://latex-beamer.sourceforge.net/
%% idea contributed by H. Turgut Uyar
%% template based on a template by Till Tantau
%% this template is still evolving - it might differ in future releases!

\documentclass{beamer}
\usepackage[brazil]{babel}
\usepackage[utf8]{inputenc}
\usepackage{amsfonts}
\usepackage{amsmath}

\mode<presentation>
{
    \usetheme{PaloAlto}
    
    \setbeamercovered{transparent}
}


\title{Sistema TRUE}
%\subtitle{}

% - Use the \inst{?} command only if the authors have different
%   affiliation.
\author{Danilo Ávila e Tales Porto}
%\author{\inst{1}}

% - Use the \inst command only if there are several affiliations.
% - Keep it simple, no one is interested in your street address.
\institute[UnB]
{
    %\inst{1}%
    Departamento de Ciência da Computação\\
    Instituto de Ciências Exatas\\
    Universidade de Brasília
}

\date{07 de dezembro de 2011}


% This is only inserted into the PDF information catalog. Can be left
% out.
%\subject{Talks}



% If you have a file called "university-logo-filename.xxx", where xxx
% is a graphic format that can be processed by latex or pdflatex,
% resp., then you can add a logo as follows:

% \pgfdeclareimage[height=0.5cm]{university-logo}{university-logo-filename}
% \logo{\pgfuseimage{university-logo}}



% Delete this, if you do not want the table of contents to pop up at
% the beginning of each subsection:
%\AtBeginSubsection[]
%{
%\begin{frame}<beamer>
%    \frametitle{Sumário}
%    \tableofcontents[currentsection,currentsubsection]
%    \end{frame}
%}

% If you wish to uncover everything in a step-wise fashion, uncomment
% the following command:

%\beamerdefaultoverlayspecification{<+->}

\begin{document}

% ------------- TITLE PAGE -------------
\begin{frame}
\titlepage
\end{frame}
% ------------- TITLE PAGE -------------


% ------------- SUMARIO -------------
\begin{frame}
	\frametitle{Sumário}
	\tableofcontents
\end{frame}


% ------------- Introdução -------------
\section{Introdução}

% ------------- Contexto -------------
\subsection{Contexto}

\begin{frame}
    \frametitle{\ldots}
\end{frame}

% ------------- Problema -------------

% ------------- Justificativa -------------

% ------------- Sobre o trabalho -------------

% ------------- Hipotese -------------

% ------------- Objetivos -------------

% ------------- Metodologia -------------

% ------------- Resultados Esperados -------------

% ------------- Cronograma -------------


% ------------- REFERENCIAS -------------

\nocite{fabriciobuzzeto,weiser2,saocarlos,yang,hewitt,violajones}

\section{Referências}

\frame[allowframebreaks]{
  \frametitle{Referências}
  \bibliographystyle{plain}
  \bibliography{bibliografia}
}

\begin{frame}
    \frametitle{ }
    \centerline{Obrigado!}
\end{frame}

\end{document}