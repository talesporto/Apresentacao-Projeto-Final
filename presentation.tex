%% LaTeX Beamer presentation template (requires beamer package)
%% see http://latex-beamer.sourceforge.net/
%% idea contributed by H. Turgut Uyar
%% template based on a template by Till Tantau
%% this template is still evolving - it might differ in future releases!

\documentclass{beamer}
\usepackage[brazil]{babel}
\usepackage[utf8]{inputenc}
\usepackage{amsfonts}
\usepackage{amsmath}

\mode<presentation>
{
    \usetheme{PaloAlto}
    
    \setbeamercovered{transparent}
}


\title{Sistema TRUE}
%\subtitle{}

% - Use the \inst{?} command only if the authors have different
%   affiliation.
\author{Danilo Ávila e Tales Porto}
%\author{\inst{1}}

% - Use the \inst command only if there are several affiliations.
% - Keep it simple, no one is interested in your street address.
\institute[UnB]
{
    %\inst{1}%
    Departamento de Ciência da Computação\\
    Instituto de Ciências Exatas\\
    Universidade de Brasília
}

\date{07 de dezembro de 2011}


% This is only inserted into the PDF information catalog. Can be left
% out.
%\subject{Talks}



% If you have a file called "university-logo-filename.xxx", where xxx
% is a graphic format that can be processed by latex or pdflatex,
% resp., then you can add a logo as follows:

% \pgfdeclareimage[height=0.5cm]{university-logo}{university-logo-filename}
% \logo{\pgfuseimage{university-logo}}



% Delete this, if you do not want the table of contents to pop up at
% the beginning of each subsection:
%\AtBeginSubsection[]
%{
%\begin{frame}<beamer>
%    \frametitle{Sumário}
%    \tableofcontents[currentsection,currentsubsection]
%    \end{frame}
%}

% If you wish to uncover everything in a step-wise fashion, uncomment
% the following command:

%\beamerdefaultoverlayspecification{<+->}

\begin{document}

% ------------- TITLE PAGE -------------
\begin{frame}
\titlepage
\end{frame}
% ------------- TITLE PAGE -------------


% ------------- SUMARIO -------------
\begin{frame}
	\frametitle{Sumário}
	\tableofcontents
\end{frame}


% ------------- Introdução -------------
\section{Introdução}

	\subsection{Computação Ubíqua}
		\begin{frame}
	    	\frametitle{Computação Ubíqua}
		\end{frame}
		
	\subsection{SmartSpace}
		\begin{frame}
	    	\frametitle{SmartSpace}
		\end{frame}
		
	\subsection{UnBquitous}
		\begin{frame}
	    	\frametitle{UnBquitous}
		\end{frame}

% ------------- Problema -------------
\section{Problema}

	\begin{frame}
    	\frametitle{Problema}
    	Qual a melhor forma do middleware conhecer a identidade dos usuários
    	presentes no SmartSpace?
	\end{frame}

% ------------- Hipoteses e objetivos -------------
\section{Hipóteses e objetivos}

	\begin{frame}
    	\frametitle{Hipóteses e objetivos}
    	Acreditando em um sensor, relativamente novo, denomidado \textif{Kinect} e
    	na confiabilidade do \textit{Eigenfaces}, algoritmo de reconhecimento
    	facial, objetivamos desenvolver um sistema que rastreasse e reconhecesse os
    	usuários presentes no SmartSpace provendo ao middleware informações de
    	identificação e localização. A esse sistema foi dado o nome de TRUE.
	\end{frame}
	
% ------------- Sistema TRUE -------------
\section{Sistema TRUE}

	\begin{frame}
    	\frametitle{Sistema TRUE}
    	O sistema TRUE se divide em 4 modulos:
    		\begin{itemize}
    		  \item \textbf{Rastreamento}
    		  \item \textbf{Reconhecimento}
    		  \item \textbf{Registro}
    		  \item \textbf{Integração}
    		\end{itemize}
    \end{frame}
    
	% ------------- Sistema TRUE -> Rastreamento -------------
    \subsection{Rastreamento}
		\begin{frame}
	    	\frametitle{Rastreamento}
	    	
	    \end{frame}
    
	% ------------- Sistema TRUE -> Reconhecimento -------------
    \subsection{Reconhecimento}
		\begin{frame}
	    	\frametitle{Reconhecimento}
	    	
	    \end{frame}
    
	% ------------- Sistema TRUE -> Registro -------------
    \subsection{Registro}
		\begin{frame}
	    	\frametitle{Registro}
	    	
	    \end{frame}
    
	% ------------- Sistema TRUE -> Integração -------------
    \subsection{Integração}
		\begin{frame}
	    	\frametitle{Integração}
	    	
	    \end{frame}
    
	% ------------- Sistema TRUE -> Relação entre os modulos -------------
    \subsection{Relação entre os modulos}
		\begin{frame}
	    	\frametitle{Relação entre os modulos}
	    	
	    \end{frame}

% ------------- Resultados obtidos -------------
\section{Resultados obtidos}

% ------------- Conclusão -------------
\section{Conclusão}

\nocite{fabriciobuzzeto,weiser2,saocarlos,yang,hewitt,violajones}

% ------------- Referências -------------
\section{Referências}

\frame[allowframebreaks]{
  \frametitle{Referências}
  \bibliographystyle{plain}
  \bibliography{bibliografia}
}

\begin{frame}
    \frametitle{ }
    \centerline{Obrigado!}
\end{frame}

\end{document}